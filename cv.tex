%!TEX TS-program = xelatex
\documentclass[]{friggeri-cv}
\usepackage{pifont}
\usepackage{graphicx}
\addbibresource{bibliography.bib}

\begin{document}
\header{JUNJIE}{ PAN}
       {Machine Learning and Software Engineering}


% In the aside, each new line forces a line break
\begin{aside}
  \section{Address}
    37 Mill Road
    Cambridge
    United Kingdom
    CB1 2AB
    ~
  \section{Contact}
  	\href{https:uk.linkedin.com/in/jeffpanuk}{Linkedin: jeffpanuk}
   \raisebox{-0.1cm}{\scalebox{1.5}{\ding{41}}}\href{mailto:kevinjjp@gmail.com}{kevinjjp@gmail.com}
    \raisebox{-0.1cm}{\scalebox{1.5}{\ding{41}}}  \href{mailto:ahhs_pjj@163.com}{ahhs\_pjj@163.com}
    \raisebox{-0.1cm}{\scalebox{1.5}{\ding{38}}} (+44)07835232266
	 \raisebox{-0.1cm}{\scalebox{1.5}{\ding{38}}} (+86)18355999722
	~
  \section{Languages}
    English (professional)
    Chinese (native)
    ~
  \section{Programming}
    {\color{red} $\varheartsuit$} Python, C
    Matlab
    Bash
    VHDL, Verilog
    Latex
    ~
  \section{Professional Skills}
	Machine Learning
	Speech Recognition
	Machine Translation
	Dialogue System
	DNN, HMM, GMM
	Data Mining
	Signal Processing
	NLP, HTK
  	Linux/Unix, Github
  	FPGA, Arduino, Robotics  
  	~
  \section{Certification}
  	CCNA
  	IELTS(7.5)	
\end{aside}

\section{Summary}

I am interested in machine learning, speech, natural language processing, and hardware programming. I have been studying in UK from 2013 till now, and have experience in machine learning and speech area for one year, in electronics and hardware programming for four years. I always act as leader in group work, and can efficiently co-operate with different people.

\section{Education}

\begin{entrylist}
  \entry
    {Since 2015}
    {University of Cambridge}
    {Cambridge, United Kingdom}
    {Master's Degree, Machine Learning, speech and Language Technology\\
    \emph{Will graduate in the mid of August}}
  \entry
    {2013–2015}
    {University of Birmingham}
    {Birmingham, United Kingdom}
    {Bachelor's Degree, Electronic, Electrical and System Engineering\\\emph{Graduated with Honours, Class I}}
  \entry
    {2011–2013}
    {Huazhong University of Science and Technology}
    {Wuhan, China}
    {Bachelor's Degree, Information Engineering\\\emph{Participated in 2+2 program to University of Birmingham in 2013}}
\end{entrylist}    
\section{Academic Experience}
\begin{entrylist}
  \entry
    {04/2016-\\(on-going)}
    {Automated Language Teaching and Assessment (ALTA)}
    {Cambridge}
    {\textsf{Description:}\\
    ALTA institute is founded by Cambridge English Language Assessment to conduct research  in automated assessment of textual and spoken materials.\medskip \\
    My research in ALTA is to improve the adaptation performance of ASR systems for non-native speakers with different first languages using unsupervised/semi-supervised learning}
\entry
    {10/2015-\\04/2016}
    {Kaggle Competition}
    {Cambridge}
    {\emph{There were 3 tasks: Regression, Classification and Density Modelling. I was responsible for the first two.}\\
    \textsf{My Work:}\hfill{\footnotesize\addfontfeature{Color=lightgray} Supervisor: Prof. Rich Turner}
    \begin{itemize}
    \item Regression: \hfill{\footnotesize\addfontfeature{Color=lightgray} Final Rank: 7th}
    \begin{itemize}
	   \item[-] Implement different missing data imputation methods.
	   \item[-] Build Gaussian process, Bayesian ridge regression, nearest neighbour, decision trees, kernel ridge, and support vector regression models and evaluate their performance.
	   \item[-] Construct regression model with the best configuration.
    \end{itemize}
    \item Classification:\hfill{\footnotesize\addfontfeature{Color=lightgray} Final Rank: 5th}
	\begin{itemize}
	\item[-] Apply PCA to given dataset. 
	\item[-] Train k-nearest neighbours, neural networks, Gaussian process and support vector machines, and investigate their performance.
	\item[-] [-] Construct classification model with the best configuration.
	\end{itemize}   
    \end{itemize}}     
\end{entrylist}
\begin{entrylist}
\entry
    {10/2015-\\04/2016}
    {Large Vocabulary Speech Recognition}
    {Cambridge}
    {\emph{This project aims to investigate three parts of state-of-art large vocabulary speech recognition: language modelling, acoustic model speaker adaptation and system combination.}\\
    \textsf{My Work:}\hfill{\footnotesize\addfontfeature{Color=lightgray}Supervisor: Prof. Mark Gales}
    \begin{itemize}
    \item Use EM algorithm to estimate language model(LM) interpolation weights and combine five provided LMs.
    \item Use HTK(version 3.5) to implement acoustic model cross-adaptation among plp, grph-plp, tandem, grph-tandem and hybrid systems. 
    \item Achieve ROVER combination and Confusion Network Combination using dynamic programming. 
    \item Analyse results and build the final version of evaluation system for testing dataset.
\end{itemize}
}   
  \entry
	{11/2014-\\03/2015}
    {Interactive Clothing - Smart Hoodies}
    {Birmingham}
    {\emph{This project aims to design an smart hoodie that can monitor user's daily exercise, give suggestions and be controlled by smart phones}\\
    \textsf{My Work:}\hfill{\footnotesize\addfontfeature{Color=lightgray} Supervisor: Prof. Chris Baber}
    \begin{itemize}
    \item Hardware design: layout of electronics, arrangement of wiring, and power supply system with consideration of comfort, safety and reliability.
    \item Algorithm design: real-time monitoring, dynamic thresholds, negative feedback and reinforcement learning.
    \item Controlling APP design: an APP on Android system to control the smart hoodie via Bluetooth.
	\end{itemize}
	}
	
  \entry
    {11/2013-\\05/2014}
    {Auto-Tracking Robot Competition}
    {Birmingham}
    {\emph{This project aims to develop an auto-tracking three-wheel robot with high speed and stability}\\
    \textsf{My Work:}\hfill{\footnotesize\addfontfeature{Color=lightgray}Supervisor: Mr. Phil Atkins}
    \begin{itemize}
    \item Allocate work to each group members and design the timetable
    \item Robot design: robot structure and sensors layout
    \item Hardware programming: high-accuracy tracking with negative feedback controlling
    \end{itemize}
}

\end{entrylist}
\section{Working Experience}

\begin{entrylist}
  \entry
    {07/2015-\\09/2015}
    {Anhui Branches of China Mobile Group Design Institute}
    {Anhui, China}
    {\textbf{Position:} Network Designer(Internship)\\
    \textbf{Resposibility:} Taking part in the program of designing the Stage-3 4G wireless network in Anhui province in China. Learning the relevant practical knowledge about large-scale network configuration, construction of base station. Also, participating the work in field test, data analysis and network optimization.}
    
\end{entrylist}

\section{Referee}
    \textbf{Professor Bill Byrne}
    \\\raisebox{-0.05cm}{\scalebox{1.5}{\ding{41}}}\href{mailto:bill.byrne@eng.cam.ac.uk}	      {bill.byrne@eng.cam.ac.uk} \raisebox{-0.05cm}{\scalebox{1.5}{\ding{37}}} \href{}{+44(0)1223 332651}
    \\Course Director of Machine Learning, Speech and Language Technology
    \\Department of Engineering, University of Cambridge\\
    
    \textbf{Professor Chris Baber}\\ 
    \raisebox{-0.1cm}{\scalebox{1.5}{\ding{41}}}\href{mailto:C.BABER@bham.ac.uk}{C.BABER@bham.ac.uk} \raisebox{-0.05cm}{\scalebox{1.5}{\ding{37}}} \href{}{+44(0)121 414 3965}
    \\Chair of Pervasive and Ubiquitous Computing\\Department of Engineering, University of Birmingham

%%% This piece of code has been commented by Karol Kozioł due to biblatex errors. 
% 
%\printbibsection{article}{article in peer-reviewed journal}
%\begin{refsection}
%  \nocite{*}
%  \printbibliography[sorting=chronological, type=inproceedings, title={international peer-reviewed conferences/proceedings}, notkeyword={france}, heading=subbibliography]
%\end{refsection}
%\begin{refsection}
%  \nocite{*}
%  \printbibliography[sorting=chronological, type=inproceedings, title={local peer-reviewed conferences/proceedings}, keyword={france}, heading=subbibliography]
%\end{refsection}
%\printbibsection{misc}{other publications}
%\printbibsection{report}{research reports}

\end{document}
